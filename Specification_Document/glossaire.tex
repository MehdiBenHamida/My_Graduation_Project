\chapter*{Glossaire}


\textbf{UML : } Unified Modeling Language \\

\textbf{RSFTA : } Service Réseau de Services Fixes de Télécommunications Aéronautiques \\

\textbf{BNI : } Bureau NOTAM International\\

\textbf{PIB : } Bulletins d’Information Pré-vol\\

\textbf{ADAMS : } Aeronautical Data And Message Handling System\\

\textbf{BIA : } Bureaux d’Information Aéronautique \\