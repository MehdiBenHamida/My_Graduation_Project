\chapter*{Conclusion Générale}
\addcontentsline{toc}{chapter}{Conclusion Générale}
%Rappel du context
Le travail élaboré dans le présent rapport a été effectué dans le cadre d’un projet développé au service DNA à l’OACA au cours du stage d’immersion en entreprise. Ce stage a pour but d’intégrer les élèves ingénieur  à l’ENSI dans le monde d’entreprise à la fin de la deuxième année. Il vise aussi de les préparer pour le projet de fin d’étude qu’ils auront à la fin de leur cursus universitaire.\\

Nous avons présenté dans ce rapport les différentes étapes menant à la conception et le développement d’une solution pour l’apurement de la base de données du système STANOS. \\

Dans le présent rapport, une étude afférente du sujet a été menée. Cette étude se décompose essentiellement en cinq chapitres. Dans le premier chapitre, nous avons présenté l’organisme d’accueil dans lequel nous avons effectué notre stage d’immersion en entreprise. On a présenté aussi dans ce chapitre le réseau RSFTA sur lequel porte notre projet de stage. Dans un deuxième chapitre, nous avons illustré l’état de l’art, nous avons aussi réalisé une étude de l’existant ainsi que le contexte du travail réalisé. Dans un troisième chapitre, nous avons spécifié les besoins qui doivent être précisé les différentes fonctionnalités du système et les services qui doivent les satisfaire. Dans le quatrième chapitre, nous avons déterminé la conception de l’application selon l’architecture MVC adoptée. Dans cette phase, nous avons présenté le diagramme de classes et le diagramme d’entité relation de la base de données du système développé. Pour le dernier chapitre intitulé « Réalisation », nous avons illustré l’aspect pratique de notre travail. En effet, nous avons défini l’environnement matériel et logiciel de notre réalisation. Dans une deuxième partie de ce chapitre, nous avons illustré dans des captures d’écran l’exécution de notre travail. \\

Tout au long de ce stage, nous avons essayé d’implémenter plusieurs fonctionnalités, tel que la gestion des rôles des utilisateurs, et la manipulation des bases de données en utilisant différentes technologies. Nous avons pu, alors, développer nos compétences techniques aussi bien que nos compétences soft au sein de l'entreprise (se présenter formellement, respect du temps, communiquer et se faire défendre ses idées, etc.. )\\

Enfin, nous pouvons conclure qu’il devrait toujours réfléchir à la contrainte de temps réel dans les systèmes de communication et savoir que la lenteur de transfert des messages peut mener à des situations critiques voir des catastrophes. C’est pour cette raison, il faut prendre en compte cette contrainte dans la conception de les bases de données des futurs systèmes pareils et prévoir une mise à jour automatique.  \\
