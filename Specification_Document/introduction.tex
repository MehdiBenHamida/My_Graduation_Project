\chapter*{Introduction générale} % Pas de numérotation
\addcontentsline{toc}{chapter}{Introduction générale}


Depuis l’invention du premier avion en 1903 appelé Wright Flyer, la gestion de trafic aérien fut un des défis les plus difficiles à résoudre vu le nombre des vols ascendants. Cette gestion a pour but de maintenir en sécurité le trafic aérien, d’ou l’apparition des organismes internationaux qui s’occupent de cette tâche. L’OACI, Office d’Aviation Civile International est l’organisme qui se charge d’établir les protocoles standardisés de de communication entre les différents aérodromes partout dans le monde. Il normalise aussi les types de messages échangés ainsi que les canaux de transfert utilisés. L’OACA, Office des Aéroport et d’Aviation Civile, qui dérive de l’OACI, est l’organisme nationale qui s’occupe de l’espace aérien tunisien. Sa tâche est de s'occuper de tout ce qui est en relation avec les aéroports internationaux existants sur le territoire tunisien. Cet organisme est décomposé en plusieurs divisions, parmi lesquelles on cite; le centre de contrôle régional. \\

Dans le domaine du contrôle aérien, un centre de contrôle régional, est un centre assurant la sécurité du trafic aérien. L’échange de messages aéronautiques entre ces centres assure l’exécution sûre, rapide et efficace des vols. Il consiste aussi à accélérer et ordonner la circulation aérienne. C’est pour cela que la disponibilité des liaisons entre ces centres s’avère indispensable décisif vu son importance dans la sécurité de la navigation aérienne, la régularité des vols, et le traitement des messages aéronautiques. \\

La diponibilité des messages aéronautiques est un affaire indispensable. Dans ce cadre, s'inscrit notre projet de stage d'immersion en entreprise qui a pour but la conception et la réalisation d'une solution d'apurement de la base de données du système STANOS.\\

Le présent rapport s'organise autour de cinq chapitres:\\
Le premier chapitre intitulé {\bf "Présentation générale"} comprend une présentation de l'organisme dans lequel nous avons effectué notre stage. Il comprend aussi une définition du système autour duquel se déroule notre stage. \\
Le deuxième chapitre intitulé {\bf "Étude préalable"} comprend une revue théorique des notions sur lesquelles s’appuie notre projet en une étude de l’existant, des critiques et une proposition solution à notre problématique.\\
Le troisième chapitre intitulé  {\bf "Analyse et spécification des besoins"} expose la spécification des besoins fonctionnels et non fonctionnels de l’application.\\
Un quatrième chapitre intitulé  {\bf "Conception"} met en évidence la modélisation conceptuelle de l’application.\\
Le dernier chapitre intitulé  {\bf "Réalisation"} inclut une présentation de l’environnement de travail avec la description de quelques interfaces de l’application.\\

Enfin, nous clôturons ce rapport par une conclusion générale dans laquelle nous résumons notre solution et exposons quelques perspectives futures.\\